\documentclass[12pt]{article}

\usepackage{enumerate}
\usepackage{amsmath}
\usepackage{amsthm}
\usepackage{amssymb}
\usepackage{changepage}
\usepackage{graphicx}

\title{Worksheet 1: SMP}
\author{Rin Meng 51940633\\ Kevin Zhang 10811057\\ Mika Panagsagan 29679552}
\date{\today}

\begin{document}

\maketitle

\textbf{Step 3: Identify similar problems. What are the similarities?}
   \begin{enumerate}
    \item Marriage Problem (Stable Marriage Problem), where each men has their own primary choice,
    each women has their own primary choice, and each person has a preference list.
    \item Admission-Student problem, where each student has their own primary choice for colleges,
    and each college has their own primary choice for students, each party also has a preference list.
   \end{enumerate}

\textbf{Step 4: Evaluate simple algorithmic approaches, such as brute force.}
   \begin{enumerate}
    \item Brute force algorithm
        \begin{enumerate}
            \item Each student has a preference list of employers, 
            and each employer has a preference list of students.
            \item Employer-optimal choice: For all employers $E = {e_1, e_2, \ldots, e_n}$ we will access their
            preference list $P(e_i)$, add all the match to another list $M$.
            looping through all of the students $S = {s_1, s_2, \ldots, s_n}$ and checking each of their own
            preference list $P(s_i)$ add all the match to another list $M$.
            \item Student-optimal choice: For all students $S = {s_1, s_2, \ldots, s_n}$ we will access their
            preference list $P(s_i)$, and match it by
            looping through all of the employers $E = {e_1, e_2, \ldots, e_n}$ and checking each of their own
            preference list $P(e_i)$, add all the match to another list $M$.
        \end{enumerate}
    
   \end{enumerate}

\end{document}
