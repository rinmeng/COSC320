\documentclass[12pt]{article}

\usepackage{enumerate}
\usepackage{amsmath}
\usepackage{amsthm}
\usepackage{amssymb}
\usepackage{changepage}
\usepackage{graphicx}

\title{Worksheet 1: SMP}
\author{Rin Meng 51940633\\ Kevin Zhang 10811057\\ Mika Panagsagan 29679552}
\date{\today}

\begin{document}

\maketitle

\section{4. Evaluate simple algorithmic approaches, such as brute force.}

\begin{enumerate}
    \item Bound by the time $t(n)$ we can say that our worst-case scenario of our brute force algorithm is
    $O(n!)$.
        \begin{enumerate}[-]
            \item The number of ways to match $n$ employers and $n$ students is $n!$.
            \item For each employer, we must check each student's preference list and 
            for each student, we must check each employer's preference list, and that is $O(n^2)$.
            \item This results in a worst-case time complexity of $O(n! \cdot n^2)$.
            \item Since $n$! grows faster than $n^2$, we can simplify this to $O(n!)$.
        \end{enumerate}
    \item Each potential solution is a perfect matching between $n$ employers and $n$ students, therefore the total
    number of potential solutions is $n$!.
    \item Overall worst-case running time of the brute force will always be $O(n!)$.
        \begin{enumerate}[-]
            \item We generate $n$! potential solutions.
            \item For each solution, we spend $O(n^2)$ time to check if it is stable.
            \item Since generating a solution takes negligible time, compared to checking it, the dominant 
            term is 
            \[ O(n! \cdot n^2) = O(n^2) \]
            \item Since $n$! grows faster than $n^2$, we can simplify this to $O(n!)$.
        \end{enumerate}
    Therefore, the brute force algorithm has a factorial time complexity of $O(n!)$, which is extremely inefficient for
    large $n$.
\end{enumerate}

\section{Design a better algorithm.}
    We can do this by using the Gale-Shapley algorithm.
    \begin{enumerate}
        \item Input 
            \begin{enumerate}
                \item Two preference lists, one for students and one for employers.
                \item The number of students and employers, $n$.
            \end{enumerate}
        \item Steps:
            \begin{enumerate}
                \item Initialization: Create an empty list of matched pairs. 
                All applicants are initially unmatched, and all 
                employers are initially unmatched.
                \item Proposal Phase: Each unmatched applicant proposes to the 
                first employer on their preference list who has not already rejected them.
                \item Employer's response:
                    \begin{enumerate}
                        \item Each employer receives proposals and considers them:
                        \begin{enumerate}[-]
                            \item If they are unmatched, they accept the proposal.
                            \item If they are already matched but prefer 
                            the new applicant over their current match, 
                            they reject the current match and accept the new proposal.
                            \item If they prefer their current match, they reject the new proposal.
                        \end{enumerate}
                    \end{enumerate}
                \item Repeat: Applicants who have been rejected by all employers or who haven't 
                yet been matched will propose to the next employer on their list.
                \item Termination: The algorithm terminates when no 
                applicants are left to propose or when everyone is matched. 
                At this point, we have a stable matching.
            \end{enumerate}
        \item Time complexity: 
            \begin{enumerate}
                \item Time per proposal, each student proposes to at most $n$ employers, 
                and each employer receives at most $n$ proposals, so the time per proposal is $O(n)$.
                \item Total time complexity, since there are $n$ students and $n$ employers,
                the total time complexity is $O(n^2)$.
            \end{enumerate}
        \item Walkthrough:
            \begin{enumerate}
                \item Let $n = 3$, $S$ be the student set, and $E$ be the employer set,
                 $M$ be the matching set, and $P$ be the preference list.
                \item $S = \{s_1, s_2, s_3\}$ and $E = \{e_1, e_2, e_3\}$, $M = \emptyset$.
                \item $P(s_1) = \{e_1, e_2, e_3\}$, $P(s_2) = \{e_2, e_1, e_3\}$, $P(s_3) = \{e_3, e_2, e_1\}$.
                \item $P(e_1) = \{s_1, s_2, s_3\}$, $P(e_2) = \{s_2, s_3, s_1\}$, $P(e_3) = \{s_3, s_1, s_2\}$.
                \begin{enumerate}
                    \item $s_1$ proposes to $e_1$, $e_1$ accepts.
                    \item $s_2$ proposes to $e_2$, $e_2$ accepts.
                    \item $s_3$ proposes to $e_3$, $e_3$ accepts.
                \end{enumerate}
                \item Terminate: we the most stable matching, which priotiizes the students.
                \[M = \{(s_1, e_1), (s_2, e_2), (s_3, e_3)\}\]
            \end{enumerate}
        \item Challenges:
            \begin{enumerate}
                \item Let $n = 3$, $S$ be the student set, and $E$ be the employer set,
                 $M$ be the matching set, and $P$ be the preference list.
                \item $S = \{s_1, s_2, s_3\}$ and $E = \{e_1, e_2, e_3\}$, $M = \emptyset$.
                \item $P(s_1) = \{e_2, e_3\}$, $P(s_2) = \{e_1, e_3\}$, $P(s_3) = \{e_2, e_1\}$.
                \item $P(e_1) = \{s_1, s_2, s_3\}$, $P(e_2) = \{s_2, s_3, s_1\}$, $P(e_3) = \{s_3, s_1, s_2\}$.
                \begin{enumerate}
                    \item $s_1$ proposes to $e_2$, $e_2$ accepts.
                    \item $s_2$ proposes to $e_1$, $e_1$ accepts.
                    \item $s_3$ proposes to $e_2$, $e_2$ accepts.
                    \[ M = \{(s_1, e_2), (s_2, e_1), (s_3, e_2)\} \]
                    \item Now we have to goto the employer's preference list, since there are two
                    primary choice for student $s_1$ and $s_3$, we have to reject one of them.
                    \item $e_1$ stays the same, goes with $s_2$.
                    \item $e_3$ rejects $s_1$ and accepts $s_2$.
                    \item $e_2$ rejects $s_2$ and accepts $s_3$.
                    \[ M = \{(s_1, e_2), (s_2, e_1), (s_3, e_3)\} \]
                \end{enumerate}
            \end{enumerate}
            
    \end{enumerate}



\end{document}
