\documentclass[12pt]{article}

\usepackage{enumerate}
\usepackage{amsmath}
\usepackage{amsthm}
\usepackage{amssymb}
\usepackage{changepage}
\usepackage{graphicx}

\title{Worksheet 1: SMP}
\author{Rin Meng 51940633\\ Kevin Zhang 10811057\\ Mika Panagsagan 29679552}
\date{\today}

\begin{document}

\maketitle

\section{Build intuition through examples.}
    From this point onwards, we will read $\textbf{Prefers}(x, y)$ as $x$ prefers $y$ over some other option.

    \begin{enumerate}
        \item Small and trivial instances of the problem, consider these instances:
            \begin{enumerate}
                \item Only one student $s$ and one employer $e$:
                \begin{enumerate}[-]
                    \item Students: ${s_1}$
                    \item Employers: ${e_1}$
                    \item Preferences: 
                        \begin{enumerate}
                            \item $\textbf{Prefers}(s_1, e_1)$
                            \item $\textbf{Prefers}(e_1, s_1)$
                        \end{enumerate}
                    \item Triviality: Only one student and one employer, so there is only one match.
                \end{enumerate}
                \item Equal preferences for all students and employers:
                \begin{enumerate}
                    \item Students: ${s_1, s_2}$
                    \item Employers: ${e_1, e_2}$
                    \item Preferences:
                        \begin{enumerate}
                            \item $\textbf{Prefers}(s_1, e_1 \lor e_2)$
                            \item $\textbf{Prefers}(s_2, e_1 \lor e_2)$
                            \item $\textbf{Prefers}(e_1, s_1 \lor s_2)$
                            \item $\textbf{Prefers}(e_2, s_1 \lor s_2)$
                        \end{enumerate}
                    \item Triviality: All students and employers have indifferent preferences, 
                    so there are multiple stable matchings.
                \end{enumerate}
                \item Perfectly matched preferences between students and employers:
                \begin{enumerate}
                    \item Students: ${s_1, s_2}$
                    \item Employers: ${e_1, e_2}$
                    \item Preferences:
                        \begin{enumerate}
                            \item $\textbf{Prefers}(s_1, e_1)$
                            \item $\textbf{Prefers}(s_2, e_2)$
                            \item $\textbf{Prefers}(e_1, s_1)$
                            \item $\textbf{Prefers}(e_2, s_2)$
                        \end{enumerate}
                    \item Triviality: All are perfectly matched, therefore there is only one stable matching.
                \end{enumerate}
                \end{enumerate}
        \item Potential solutions to these instances:
            \begin{enumerate}
                \item Only one student $s$ and one employer $e$:
                \begin{enumerate}[-]
                    \item The only stable matching is $(s_1, e_1)$.
                    \item This solution is trivial but is optimal as well.
                \end{enumerate}
                \item Equal preferences for all students and employers:
                \begin{enumerate}[-]
                    \item There are multiple stable matchings: 
                    $(s_1, e_1), (s_2, e_2)$ and $(s_1, e_2), (s_2, e_1)$.
                    \item This solution is optimal, but not unique.
                \end{enumerate}
                \item Perfectly matched preferences between students and employers:
                \begin{enumerate}[-]
                    \item The only stable matching is $(s_1, e_1), (s_2, e_2)$.
                    \item This solution is optimal and unique.
                \end{enumerate}
            \end{enumerate}
            There are many ways to conclude if a solution is better.
            When we consider fairness and satisfaction, we may value $\textbf{Instance b}$ more than 
            others. However, when we consider uniqueness and optimality, 
            we may value $\textbf{Instance c}$ more than others.
    \end{enumerate}
\section{Developing a Formal Problem Specification}
   \begin{enumerate}
    \item Notation for describing the problem instance.
        \begin{enumerate}
            \item Let $S = \{s_1, s_2, \ldots, s_n\}$ 
            be the set of students.
            \item Let $E = \{e_1, e_2, \ldots, e_n\}$ be 
            the set of employers.
            \item Student's preference list $P(s_i)$, which is a ranked 
            list of employers from most preferred to least preferred.
            \item Employer's preference list $P(e_j)$, which is a ranked
            list of students from most preferred to least preferred.
            \item A matching is a bijection $M : S \rightarrow E$, or a set of pairs $M = \{(s, e)\}$ where $s$ and $e$ 
            are matched uniquely.
            \item A blocking pair $(s_i, e_j)$ is a pair of
             student-employer that prefer each other over their current match.
                \begin{enumerate}
                    \item $s_i$ prefers $e_j$ over $M(s_i)$.
                    \item $e_j$ prefers $s_i$ over $M(e_j)$.
                \end{enumerate}
        \end{enumerate}
    \item Notation for describing a potential solution
        \begin{enumerate}
            \item A set of potential matches $M = \{(s_1, e_1), (s_2, e_2)\}$ 
            of student-employer pairs.
            \item Valid if and only if every student and employer is assigned exactly one (uniqueness).
        \end{enumerate}
    \item Good solutions
        \begin{enumerate}
            \item Optimality
                \begin{enumerate}
                    \item A solution is student-optimal if it provides the best possible match for every student.
                    \item A solution is employer-optimal if it provides the best possible match for every employer.
                \end{enumerate}
            \item Uniqueness
                \begin{enumerate}
                    \item If a unique stable matching exists, it is the only correct solution.
                    \item If multiple stable matchings exist, we will pick the one that maximizes a criterion (e.g., student happiness, fairness).
                \end{enumerate}
            \item Stability
                \begin{enumerate}
                    \item No blocking pairs exist in the matching.
                    \item A matching $M$ is stable if there exists no student-employer pair 
                    $(s_i, e_j)$ such that
                    \begin{enumerate}
                        \item $s_i$ prefers $e_j$ over $M(s_i)$, the employer currently matched with $s_i$.
                        \item $e_j$ prefers $s_i$ over $M(e_j)$, the student currently matched with $e_j$.
                    \end{enumerate}
                \end{enumerate}
        \end{enumerate}
   \end{enumerate}

\section{Identify similar problems. What are the similarities?}
   \begin{enumerate}
    \item Marriage Problem (Stable Marriage Problem), where each men has their own primary choice,
    each women has their own primary choice, and each person has a preference list.
    \item Admission-Student problem, where each student has their own primary choice for colleges,
    and each college has their own primary choice for students, each party also has a preference list.
   \end{enumerate}

\section{Evaluate simple algorithmic approaches, such as brute force.}
   \begin{enumerate}
    \item Brute force algorithm
        \begin{enumerate}
            \item Each student has a preference list of employers, 
            and each employer has a preference list of students.
            \item Employer-optimal choice: For all employers $E = {e_1, e_2, \ldots, e_n}$ we will access their
            preference list $P(e_i)$, add all the match to another list $M$.
            looping through all of the students $S = {s_1, s_2, \ldots, s_n}$ and checking each of their own
            preference list $P(s_i)$ add all the match to another list $M$.
            \item Student-optimal choice: For all students $S = {s_1, s_2, \ldots, s_n}$ we will access their
            preference list $P(s_i)$, and match it by
            looping through all of the employers $E = {e_1, e_2, \ldots, e_n}$ and checking each of their own
            preference list $P(e_i)$, add all the match to another list $M$.
        \end{enumerate}
    \item 
        \begin{enumerate}
            \item Bound by the time $t(n)$ we can say that our worst-case scenario of our brute force algorithm is
            $O(n!)$.
                \begin{enumerate}[-]
                    \item The number of ways to match $n$ employers and $n$ students is $n!$.
                    \item For each employer, we must check each student's preference list and 
                    for each student, we must check each employer's preference list, and that is $O(n^2)$.
                    \item This results in a worst-case time complexity of $O(n! \cdot n^2)$.
                    \item Since $n$! grows faster than $n^2$, we can simplify this to $O(n!)$.
                \end{enumerate}
            \item Each potential solution is a perfect matching between $n$ employers and $n$ students, therefore the total
            number of potential solutions is $n$!.
            \item Overall worst-case running time of the brute force will always be $O(n!)$.
                \begin{enumerate}[-]
                    \item We generate $n$! potential solutions.
                    \item For each solution, we spend $O(n^2)$ time to check if it is stable.
                    \item Since generating a solution takes negligible time, compared to checking it, the dominant 
                    term is 
                    \[ O(n! \cdot n^2) = O(n^2) \]
                    \item Since $n$! grows faster than $n^2$, we can simplify this to $O(n!)$.
                \end{enumerate}
            Therefore, the brute force algorithm has a factorial time complexity of $O(n!)$, which is extremely inefficient for
            large $n$.
        \end{enumerate}
   \end{enumerate}
\section{Design a better algorithm.}
    We can do this by using the Gale-Shapley algorithm.
    \begin{enumerate}
        \item Input 
            \begin{enumerate}
                \item Two preference lists, one for students and one for employers.
                \item The number of students and employers, $n$.
            \end{enumerate}
        \item Steps:
            \begin{enumerate}
                \item Initialization: Create an empty list of matched pairs. 
                All applicants are initially unmatched, and all 
                employers are initially unmatched.
                \item Proposal Phase: Each unmatched applicant proposes to the 
                first employer on their preference list who has not already rejected them.
                \item Employer's response:
                    \begin{enumerate}
                        \item Each employer receives proposals and considers them:
                        \begin{enumerate}[-]
                            \item If they are unmatched, they accept the proposal.
                            \item If they are already matched but prefer 
                            the new applicant over their current match, 
                            they reject the current match and accept the new proposal.
                            \item If they prefer their current match, they reject the new proposal.
                        \end{enumerate}
                    \end{enumerate}
                \item Repeat: Applicants who have been rejected by all employers or who haven't 
                yet been matched will propose to the next employer on their list.
                \item Termination: The algorithm terminates when no 
                applicants are left to propose or when everyone is matched. 
                At this point, we have a stable matching.
            \end{enumerate}
        \item Time complexity: 
            \begin{enumerate}
                \item Time per proposal, each student proposes to at most $n$ employers, 
                and each employer receives at most $n$ proposals, so the time per proposal is $O(n)$.
                \item Total time complexity, since there are $n$ students and $n$ employers,
                the total time complexity is $O(n^2)$.
            \end{enumerate}
        \item Walkthrough:
            \begin{enumerate}
                \item Let $n = 3$, $S$ be the student set, and $E$ be the employer set,
                 $M$ be the matching set, and $P$ be the preference list.
                \item $S = \{s_1, s_2, s_3\}$ and $E = \{e_1, e_2, e_3\}$, $M = \emptyset$.
                \item $P(s_1) = \{e_1, e_2, e_3\}$, $P(s_2) = \{e_2, e_1, e_3\}$, $P(s_3) = \{e_3, e_2, e_1\}$.
                \item $P(e_1) = \{s_1, s_2, s_3\}$, $P(e_2) = \{s_2, s_3, s_1\}$, $P(e_3) = \{s_3, s_1, s_2\}$.
                \begin{enumerate}
                    \item $s_1$ proposes to $e_1$, $e_1$ accepts.
                    \item $s_2$ proposes to $e_2$, $e_2$ accepts.
                    \item $s_3$ proposes to $e_3$, $e_3$ accepts.
                \end{enumerate}
                \item Terminate: we the most stable matching, which priotiizes the students.
                \[M = \{(s_1, e_1), (s_2, e_2), (s_3, e_3)\}\]
            \end{enumerate}
        \item Challenges:
            \begin{enumerate}
                \item Let $n = 3$, $S$ be the student set, and $E$ be the employer set,
                 $M$ be the matching set, and $P$ be the preference list.
                \item $S = \{s_1, s_2, s_3\}$ and $E = \{e_1, e_2, e_3\}$, $M = \emptyset$.
                \item $P(s_1) = \{e_2, e_3\}$, $P(s_2) = \{e_1, e_3\}$, $P(s_3) = \{e_2, e_1\}$.
                \item $P(e_1) = \{s_1, s_2, s_3\}$, $P(e_2) = \{s_2, s_3, s_1\}$, $P(e_3) = \{s_3, s_1, s_2\}$.
                \begin{enumerate}
                    \item $s_1$ proposes to $e_2$, $e_2$ accepts.
                    \item $s_2$ proposes to $e_1$, $e_1$ accepts.
                    \item $s_3$ proposes to $e_2$, $e_2$ accepts.
                    \[ M = \{(s_1, e_2), (s_2, e_1), (s_3, e_2)\} \]
                    \item Now we have to goto the employer's preference list, since there are two
                    primary choice for student $s_1$ and $s_3$, we have to reject one of them.
                    \item $e_1$ stays the same, goes with $s_2$.
                    \item $e_3$ rejects $s_1$ and accepts $s_2$.
                    \item $e_2$ rejects $s_2$ and accepts $s_3$.
                    \[ M = \{(s_1, e_2), (s_2, e_1), (s_3, e_3)\} \]
                \end{enumerate}
            \end{enumerate}
            
    \end{enumerate}

\end{document}
