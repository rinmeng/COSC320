\documentclass[12pt]{article}

\usepackage{enumerate}
\usepackage{amsmath}
\usepackage{amsthm}
\usepackage{amssymb}
\usepackage{changepage}
\usepackage{graphicx}
  
\title{Worksheet 3: Asymptotic Analysis}
\author{Rin Meng 51940633 \\ Kevin Zhang 10811057 \\ Mika Panagsagan 29679552 \\ Priyansh Mathur 84491356}
\date{\today}

\begin{document}

\maketitle

\setcounter{section}{1}

\section{Functions/Orders of Growth for Code}


\begin{enumerate}
    \item Finding maximum in a list 
    \[ \Omega(n) \leq \Theta(n) \leq O(n) \]
    \textbf{Justification:} Because the loop must iterate at least once through all
    the elements in the list to find the maximum, the lower bound is $\Omega(n)$,
    and the highest possible time complexity is $O(n)$.

    \item ``Median of three'' computation:
    \[ \Omega(1) \leq \Theta(1) \leq O(1) \]
    \textbf{Justification:} The function only compares three values and returns 
    the median of the three. This is a constant time operation, 
    so the time complexity is $O(1)$.

    \item Counting inversions:
    \[ \Omega(n) \leq \Theta(n \log n) \leq O(n^2) \]
    \textbf{Justification:} The lower bound is $\Omega(n)$ assuming that all first element
    of index $j$ satisfies the if statement. The upper bound is $O(n^2)$ because element
    at index $j$ may not satisfy the if statement until the last element.
\end{enumerate}

\end{document}
