\documentclass[12pt]{article}

\usepackage{enumerate}
\usepackage{amsmath}
\usepackage{amsthm}
\usepackage{amssymb}
\usepackage{changepage}
\usepackage{graphicx}
  
\title{Worksheet 3: Asymptotic Analysis}
\author{Rin Meng 51940633 \\ Kevin Zhang 10811057 \\ Mika Panagsagan 29679552 \\ Priyansh Mathur 84491356}
\date{\today}

\begin{document}

\maketitle

\setcounter{section}{2}

\section{Progress Measures for While Loops}

Assume that \verb|FindNeighboringInversion(A)| consumes an array \verb|A| and 
returns an index i such that \verb|A[i] > A[i+1]| or returns -1 if no such 
inversion exists. Let’s work out a bound on the number of iterations of
the loop below in terms of $n$, the length of the array \verb|A|.
\begin{verbatim}
Let i = FindNeighboringInversion(A)
While i >= 0:
    Swap A[i] and A[i+1]
    Set i to FindNeighboringInversion(A)
\end{verbatim}


\begin{enumerate}
    \item Give and work through two small inputs
        
    Let us define two arrays $A$ where
        \[ 
            A = [3, 1, 4, 2, 5]
        \]
        
        \begin{enumerate}
            \item since $A[i] > A[i + 1] = A[0] > A[1]$, $i = 0$
            \item Since $i \geq 0$, we swap $A[0]$ and $A[1]$, so $A =  [1, 3, 4, 2, 5]$
            \item Set $i = \verb|FindNeighboringInversion(A)|$
            \item Now we choose that the next inversion occurs at $i = 2$ since $A[2] > A[3]$
            \item Since $i \geq 0$, we swap $A[2]$ and $A[3]$, so $A = [1, 3, 2, 4, 5]$.
            \item Set $i = \verb|FindNeighboringInversion(A)|$
            \item Now we choose that the next inversion occurs at $i = 1$ since $A[1] > A[2]$
            \item Since $i \geq 0$, we swap $A[1]$ and $A[2]$, so $A = [1, 2, 3, 4, 5]$.
            \item Set $i = \verb|FindNeighboringInversion(A)|$
            \item There are no inversions, therefore  $i = -1$
            \item Since $i < 0$, we terminate the loop.
        \end{enumerate}

        \[
            A = [5, 4, 3, 2, 1]
        \]

        \begin{enumerate}
            \item since $A[i] > A[i + 1] = A[0] > A[1]$, $i = 0$
            \item Since $i \geq 0$, we swap $A[0]$ and $A[1]$, so $A =  [4, 5, 3, 2, 1]$
            \item Set $i = \verb|FindNeighboringInversion(A)|$
            \item Now we choose that the next inversion occurs at $i = 1$ since $A[1] > A[2]$
            \item Since $i \geq 0$, we swap $A[1]$ and $A[2]$, so $A = [4, 3, 5, 2, 1]$.
            \item Set $i = \verb|FindNeighboringInversion(A)|$
            \item Now we choose that the next inversion occurs at $i = 2$ since $A[2] > A[3]$
            \item Since $i \geq 0$, we swap $A[2]$ and $A[3]$, so $A = [4, 3, 2, 5, 1]$.
            \item Set $i = \verb|FindNeighboringInversion(A)|$
            \item Now we choose that the next inversion occurs at $i = 3$ since $A[3] > A[4]$
            \item Since $i \geq 0$, we swap $A[3]$ and $A[4]$, so $A = [4, 3, 2, 1, 5]$.
        \end{enumerate}

        From this, we can see that the number of iterations is very long.

    \item Define an inversion (not just a neighboring one), and prove that if an inversion exists at all,
        a neighboring inversion exists. 
        
        An inversion is defined as a pair of indices $(i, i + 1)$ such that 
        $A[i] > A[i + 1]$.
        
        Proof by contradiction: Assume that an inversion exists at index $i$ such that

        \[
            A[i] > A[i + 1]
        \]

        and there is no neighboring inversion. This implies that $A[i + 1] \leq A[i + 2]$.
        However, this contradicts the definition of an inversion. Therefore, if an inversion exists,
        a neighboring inversion must exist.

    \item Give upper-bounds and lower-bounds on the number of inversions in A.
    
        Upperbound:

        \[ O(\frac{n(n-1)}{2}) \]

        \textbf{Justification:} If the array is arranged in decreasing order, 
        then there are $n(n-1)/2$ inversions.

        Lowerbound:

        \[ \Omega(0) \]

        \textbf{Justification:} If the array is arranged in increasing order, then there are no inversions.
    
    \item Give a “measure of progress” for each iteration of the loop in terms of inversions. (I.e., how can we
    measure that we’re making progress toward terminating the loop?)

        The measure of progress is the number of inversions in the array $A$. If the number of inversions
        decreases, then we are making progress towards terminating the loop.

    \item Give an upper-bound on the number of iterations the loop could take.
    
        The upper-bound on the number of iterations the loop could take is $O(n^2)$.
    
    \item Prove that this algorithm sorts the array A (i.e., removes all inversions from the array).
    
        Proof by contradiction: Assume that the algorithm does not sort the array $A$. This implies that
        there exists an inversion in the array $A$. However, the algorithm terminates when there are no
        inversions in the array $A$. Therefore, the algorithm must sort the array $A$.

    

\end{enumerate}

\end{document}
