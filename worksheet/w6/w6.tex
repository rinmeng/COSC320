\documentclass[12pt]{article}
\usepackage{enumerate}
\usepackage{amsmath}
\usepackage{amsthm}
\usepackage{amssymb}
\usepackage{changepage}
\usepackage{graphicx}
  
\title{Worksheet 6: Clustering Completed}
\author{Rin Meng 51940633 \\ Kevin Zhang 10811057 \\ Mika Panagsagan 29679552 \\ Priyansh Mathur 84491356}
\date{\today}

\begin{document}

\maketitle

\section{Clustering Completed}

\begin{enumerate}
    \item We can start by assuming that there exists an intra-category
    lower similarity weight than the maximum inter-category edge. 
    Let the maximum inter-category edge in the final solution be $e$ 
    and the lower intra-category edge be $e \prime$
    If $e \prime$ had a lower weight than $e$,
    Then $e \prime$ would have been included by the algorithm, as 
    the algorithm processes it in decreasing order.
    By the time the algorithm processes $e \prime$ its endpoints would have already
    been merged, forming an intra-category connection.

    Therefore, the greedy algorithm would have already created an 
    intra-category connection at a higher similarity, 
    contradicting the assumption that $e \prime$ has a lower similarity than $e$.

    \item We can say that since the $C$ is already an optimal solution with inter-category
    edges, then it is true that the lower bound may be $s_{min}$ where it is the minimum
    similarity between any two points in the dataset, and the upper bound may be $s_{max}$ where
    it is the maximum similarity between any two points in the dataset.
\end{enumerate}


\end{document}
