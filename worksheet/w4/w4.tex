\documentclass[12pt]{article}

\usepackage{enumerate}
\usepackage{amsmath}
\usepackage{amsthm}
\usepackage{amssymb}
\usepackage{changepage}
\usepackage{graphicx}
  
\title{Worksheet 4: Playing with Graphs}
\author{Rin Meng 51940633 \\ Kevin Zhang 10811057 \\ Mika Panagsagan 29679552 \\ Priyansh Mathur 84491356}
\date{\today}

\begin{document}

\maketitle

\setcounter{section}{2}

\section{Diameter Algorithm}

Design an algorithm to find the \textbf{diameter} of an unweighted, 
undirected, \textbf{connected} graph. For short,
we’ll call this problem DIAMETER

\renewcommand{\theenumi}{3.\arabic{enumi}}
\begin{enumerate}
\item Trivial and Small Instances
    \begin{enumerate}[1.]
        \item \begin{enumerate}
            \item When a graph has a single node, the diameter is 0.
            \item When a graph has two nodes, there can only be one diameter.
            \item When a graph has no nodes, there is no diameter.
        \end{enumerate}
        \item We can make a graph of three nodes, $A$, $B$, and $C$, 
        where $A$ is connected to $B$ and $B$ is connected to $C$. 
        The diameter of this graph is 2. \verb|A -- B -- C| 
        The diameter of this graph is 2.
    \end{enumerate}
\item Represent the Problem
    \begin{enumerate}[1.]
        \item $G = (V, E)$ is an unweighted, undirected, connected graph.
        $V$ is the set of vertices in the graph. $E$ set of edges in the graph.
        Using the graph above, we can say, $V = \{A, B, C\}$ and $E = \{(A, B), (B, C)\}$.
        $G = (\{A, B, C\}, \{(A, B), (B, C)\})$
        \item One trivial case can be represented as $V = \{A, B\}$,
         $E = \{(A, B)\}$, $G = (\{A, B\})$
         Since we only have two nodes, the diameter is always going to be 1.
        \item 
        \item 
        Constraints for a node may not have an edge to itself: 
        \[\forall v \in V, (v,v) \notin E\]
        Constraints for an edge between two nodes may only appear once:
        \[\forall v, u \in V, (v,u) \in E \implies (v,u) = (u, v)\]    
    \end{enumerate}
\item Represent the Solution
    \begin{enumerate}[1.]
        \item 
        \item 
    \end{enumerate}
\item Similar Problems
    \begin{enumerate}[1.]
        \item Prim's Algorithm
        \item Dijkstra's Algorithm
    \end{enumerate}
\end{enumerate}


\end{document}
